%!TEX program = xelatex
% 完整编译: xelatex -> biber/bibtex -> xelatex -> xelatex
\documentclass[lang=cn,a4paper,12pt,bibend=biber]{GAN}

% \documentclass[lang=cn,a4paper,zihao=-4,bibend=bibtex,chinesefont=founder]{GAN}

\title{GAN生成对抗网络研究综述}
\author{20121034 胡才郁}
\institute{\href{https://cs.shu.edu.cn/}{计算机工程与科学学院}}

% \version{0.10}
\date{\zhtoday}


% 本文档命令
\usepackage{array}
\newcommand{\ccr}[1]{\makecell{{\color{#1}\rule{1cm}{1cm}}}}
\addbibresource[location=local]{reference.bib} % 参考文献,不要删除

\begin{document}

\maketitle

\begin{abstract}
  生成对抗网络(GAN)是无监督学习领域最近几年快速发展的一个研究方向,其主要特点是能够以一种间接的方式对一个未知分布进行建模。本文首先介绍了生成对抗网络的基本结构,列举了GAN模型优缺点;其次介绍了生成模型的评价标准以及基于GAN的各种衍生模型。近年来GAN在计算机视觉方向有着重要贡献,在图像超分辨率重构、风格迁移等都得到了应用。本文分析其研究现状和发展趋势,展开说明了每种应用的理论改进之处、优点及使用场景进行了总结,最后对于GAN的未来发展进行探讨。
  \keywords{生成对抗网络,生成模型,生成器,判别器,计算机视觉}
\end{abstract}

\section{基本GAN模型简介}

\subsection{基本GAN模型的结构}

生成对抗网络(GAN)可以拆分为两个模块:一个是\textbf{判别网络},另一个是\textbf{生成网络}。生成网络$G$生成假数据,并将其模仿成真实数据,混合到真实数据中,而辨别网络$D$的任务为把真实数据和假数据区分开。真实数据与生成的假数据的区别在于,真实数据是从真实的数据分布中采样得到的,而假数据是从生成网络$G$中采样得到的。在训练过程中,生成网络$G$的目标是尽可能地生成假数据,使得判别网络$D$无法区分真实数据和假数据;而判别网络$D$的目标是尽可能地区分真实数据和假数据。当判别网络$D$无法区分真实数据和假数据时,训练过程达到平衡,生成网络$G$即为可以完成生成任务的生成模型。

用形象的说法理解,画家(生成网络$G$)学习绘画看起来真实的图像,而画作评论家(判别网络$D$)学习区分真假图像。训练过程中,生成网络在生成逼真图像方面逐渐变强,而判别网络在辨别这些图像的能力上逐渐变强。当判别器不再能够区分真实图片和伪造图片时,训练过程达到平衡。而生成网络(艺术家)即为可以完成生成任务的生成模型。对抗与生成的过程如\figref{fig:GAN示意图}所示。

\begin{figure}[!htbp]
  \centering
  \includegraphics[width=0.7\textwidth]{image/GAN示意图.png}
  \caption[]{GAN示意图}
  \label{fig:GAN示意图}
\end{figure}

\subsection{GAN模型训练流程}

生成网络$G$通过机器生成数据,目的是学习真实样本的分布,生成相似度逼近真实样本的伪样本,而判别网络$D$的作用是区分从数据获取的真实样本和由生成网络$G$生成的伪样本。两个模型通过不断地对抗训练来迭代优化,使生成网络$G$生成的数据分布最大可能接近于真实数据分布,当判别网络$D$中,每次输出的概率基本为1/2,即判别器相当于随机猜测样本是真是假,此时说明模型达到了博弈论中的\textbf{纳什均衡},即最优状态,这就是GAN的对抗性思想。

第一阶段:固定判别器$D$,训练生成器$G$。从真实样本集中随机采样一批数据作为真实样本,从生成器生成的假样本集中随机采样一批数据作为假样本,将真实样本和假样本作为判别器的输入,计算判别器的损失函数,然后对判别器进行梯度下降更新参数。

第二阶段:固定生成器$G$,训练判别器$D$。从随机噪声分布中采样一批数据作为生成器的输入,将生成器生成的假样本作为判别器的输入,计算判别器的损失函数,然后对生成器进行梯度下降更新参数。

对于第一阶段与第二阶段不断的进行训练,直到达到预设的训练轮数,收敛后的生成网络$G$即为可以完成生成任务的生成模型。

基本GAN模型结构如\figref{fig:基本GAN模型结构}所示:
\begin{figure}[!htbp]
  \centering
  \includegraphics[width=0.55\textwidth]{image/基本GAN模型结构.png}
  \caption[]{基本GAN模型结构}
  \label{fig:基本GAN模型结构}
\end{figure}

在形式上,生成器$G$和判别器$D$的优化过程可以定义为二元极小极大博弈问题,其目标函数如\eqref{eq:二元博弈目标函数}所示:
\begin{equation}
  \min _{G} \max _{D} V(D, G) = E_{x \sim p_{\mathrm{dt}}(x)}[\log D(x)]+E_{z \sim p_{z}(x)}[\log (1-D(G(z)))]
  \label{eq:二元博弈目标函数}
\end{equation}

式中:$p_{\mathrm{dt}}$代表真实的数据分布,$p_{\boldsymbol{z}}(x)$代表生成的数据分布,训练生成器以最小化目标函数,而训练判别器以使其最大化,最终使生成器生成类似于真实数据的数据。



\subsection{生成网络}

生成器本质上是一个可微分函数,生成器接收随机变量$z$的输入,经生成器$G$后生成假样本$G(z)$。在GAN中,生成器$G$对输入变量:基本没有限制,$z$通常是一个高维的随机编码向量,可以是随机噪声或者符合某种分布的变量。生成器理论上可以逐渐学习任何概率分布,经训练后的生成网络可以生成逼真图像,但又不是和真实图像完全一样,即生成网络实际上是学习了训练数据的一个近似分布,这在数据增强应用方面尤为重要。由于其生成的样本与真实样本的分布相似,因此生成器也被称为仿真器。

\subsection{辨别网络}

判别器同生成器一样,其本质上也是可微分函数,在GAN中,判别器的主要目的是判断输入是否为真实样本,并提供反馈以指导生成器训练。判别器和生成器组成零和游戏的两个玩家,为取得游戏的胜利,判别器和生成器通过训练不断提高自己的判别能力和生成能力,游戏最终会达到一个纳什均衡状态,此时生成器学习到了与真实样本近似的概率分布,判别器已经不能正确判断输入的数据是来自真实样本还是来自生成器生成的假样本$G(z)$。即判别器每次输出的概率值都是$1/2$。

\section{GAN模型的优势}

由于生成器是一个简单的、确定的前馈网络,因此GAN可以用简单的方式对生成的数据进行采样,无需在学习过程中进行推断,因此在采样上计算较快且准确率较高。此外,GAN模型可以并行生成样本,而无需利用与输入维数成比例的运行时间。GAN能够更好建立数据分布的模型,且已广泛应用于图像处理领域,因为事实证明能够很好的与图像配合使用,能生成更清晰逼真锐利的图像。理论上,GAN由于也是神经网络,可以拟合任意一种函数,训练任何一种生成器网络,具有更加灵活的框架,与传统神经网络需要构建一个损失函数相比,GAN可以学习损失函数。
GAN具备以下优势:
\begin{enumerate}[label=\arabic*).]
  \item 能学习真实样本的分布,探索样本的真实结构,且具有更强大的预测能力;
  \item 不同于一般的机器学习模型,GAN对生成样本非常的稳定;
  \item 通过GAN生成以假乱真的样本,缓解了小样本机器学习的困难;
  \item 为指导人工智能系统完成复杂任务提供了一种全新的思路;
  \item GAN 与传统神经网络的一个重要区别是传统神经网络需要人工精心设计和建构一个损失函数,而GAN可以学习损失函数。
\end{enumerate}

正因为GAN具有上述优势,在近年来涌现出了大量的GAN模型,如DCGAN、CGAN、CycleGAN、Pix2Pix等,这些模型都是在GAN的基础上进行改进,GAN也成为了深度学习领域的一个热点。

\section{基本GAN模型的问题与挑战}

尽管GAN有许多优点,但是在实际训练GAN模型时,如何寻找纳什平衡的解是一个挑战和难点,在实际训练中容易出现判别器$D$收敛、生成器$G$发散的现象,两个网络之间很难出现很好的同步,因此GAN面临着难训练、不稳定的问题。

\subsection{训练集样本的影响}

神经网络的表现主要取决于模型自身的特点,以及训练使用的真实样本集。同样,GAN模型的训练学习的质量也受制于训练样本集的影响。

一方面,样本集的自身内在数据分布情况可能会影响 GAN的训练效率和生成质量。例如,可以定义类内距离集与类间距离集,并依此提出基于距离的可分性指数,用于量化样本可分性,并指出当不同种类样本按相同分布混合时最难以区分,使用这种样本集进行有监督学习时很难使模型有较好表现。这对于GAN的样本生成质量评价指标设计具有借鉴意义。

另一方面,GAN模型的一大特点是学习真实样本分布,因此需要足够多真实样本进行训练才能有较好表现,研究如何使用小规模训练集得到较好的GAN模型是具有挑战和意义的。GAN模型对训练集质量也有较高要求,而高质量的数据集往往难以获得,因此研究哪些数据会影响模型表现,如何规避低质量样本带来的负面影响,以降低对训练集质量的高要求,也是目前热门的研究方向。


\subsection{模式崩溃问题}

模式崩溃是指,生成器生成不了多样性的样本,而是生成了与真实样本相同的样本,这种缺陷对于数据增强而言是致命的。例如在生成人脸图片的实验中,无法生成多种风格的人脸,只能生成某一种风格的人脸。
GAN存在模式崩溃现象,当生成器$G$学习到一个参数设置,可以生成对判别器$D$而言特别逼真的样本,由于此样本很容易骗过判别器$D$,所以生成器$G$可能会一次又一次的生成相同的伪样本,最终始终生成同样的样本点,出现模式缺失,无法继续学习。与普通神经网络训练过程相比,GAN模型中存在生成器$G$与判别器$D$之间的博弈机制,这使得GAN模式崩溃问题变得复杂。

\subsection{梯度消失问题}

梯度消失也是GAN存在的问题之一,如果判别器$D$始终能够正确判断真实样本为真和生成样本为假,那么无论生成器$G$生成的样本多么好,判别器$D$都可以把它们分类为假样本,此时损失降为零,导致生成器$G$没有学习,就产生了梯度消失现象。
真实分布是一个高维分布,而生成分布来自于一个低维分布,所以其实很有可能生成分布与真实分布之间就没有重叠的部分。除此之外,不可能真正去计算两个分布,只能近似采样,所以也导致了两个分布没有重叠部分。如果辨别器$D$训练得太好,那么生成分布和原来分布基本没有重叠部分,这就导致了梯度消失;如果判别器$D$训练得不好,这样生成器得梯度又不准,就会出现错误得优化方向,因此GAN难以训练。

\subsection{收敛难度问题}

由于GAN模型的较为复杂,训练GAN模型的收敛难度很大。在GAN模型中,生成器和判别器是相互竞争的,因此在训练过程中,生成器和判别器的损失函数都是下降的,如果在此过程之中,二者的下降速度不一致,就会导致GAN模型无法收敛。进一步而言,如果生成器的损失函数下降的很快,而判别器的损失函数下降的很慢,那么生成器就会很快的学会判别器的规则,从而导致生成器生成的图像和真实图像没有区别。


\section{GAN评估方法}

对于生成问题的评估并不像评估分类、回归问题一样简单,它没有准确率、查准率等明确的指标。目前的评估方法主要是从定性评估与定量评估两个方面进行。

\subsection{定性评估}

定性评估的方法主要还是靠人的眼睛来进行判断。一般的做法是将真实图像和生成图像直接由让人来判断图像的真假,并且给出两者的相似程度,最后根据打分的结果统计一个最终的指标。
在实践中由于人的主观性是很强的,每个人的标准是不一致的,导致定性评估不是一个通用的标准。视觉检查在评估一个模型对数据的拟合程度时,在低维度数据的情况下可以工作得很好,但是在高维度数据的情况下,这种直觉性可能会导致误导。

\subsection{定量评估}

对于GAN这类生成模型而言,好的评价指标应偏向如下特点:生成样本与真实样本相似;生成样本在类内、类间保持多样化;模型在隐空间采样可控;对改变样本语义的失真和变化敏感。同时指标自身应该具有的特点包括:有明确的值域且值的大小能够反映对模型较好或较差的评价、对样本数量的需求低、计算复杂度低等。以下为两种常用于GAN生成图像的模型定量评估指标。


\subsubsection[]{Inception Score}

Inception Score是一种基于GAN生成图像的模型定量评估指标,其主要思想是通过计算生成图像的分类准确率来评估生成图像的质量。Inception Score的计算过程如下:首先将生成图像和真实图像分别输入到Inception V3网络中,得到每个图像对应的2048维特征向量,然后将这些特征向量输入到一个分类器中,得到每个图像的分类结果,最后将分类结果输入到一个评估函数中,计算生成图像的分类准确率。Inception Score的计算公式如下:
\begin{equation}
  IS = \exp\left(\frac{1}{K}\sum_{k=1}^{K} D_{KL}\left(P_{k}||\bar{P}\right)\right)
  \label{eq:inception_score}
\end{equation}

其中$N$表示生成图像的数量,$K$表示分类器的类别数,$P_{i,k}$表示第$i$张生成图像在第$k$个类别上的概率,$\hat{P}_{i,k}$表示第$i$张真实图像在第$k$个类别上的概率,$D_{KL}$表示KL散度,$IS$表示Inception Score。Inception Score的值域为$[0,+\infty]$,值越大表示生成图像的质量越好。
\subsubsection[]{Fréchet Inception Distance}

FID的主要思想是:既然预训练网络模型可以提取样本特征信息,那么分别提取真实样本与生成样本特征信息,假设特征符合多元高斯分布,再计算分布间的弗雷歇距离,距离近则生成图片质量较高,多性较好。弗雷歇距离值为下式:
\begin{equation}
  FID = \left\|\mu_{1}-\mu_{2}\right\|_{2}^{2}+\operatorname{Tr}\left(\Sigma_{1}+\Sigma_{2}-2\left(\Sigma_{1}\Sigma_{2}\right)^{1 / 2}\right)
  \label{eq:frechet_inception_distance}
\end{equation}

其中$\mu_{1}$和$\mu_{2}$分别表示真实样本和生成样本的均值,$\Sigma_{1}$和$\Sigma_{2}$分别表示真实样本和生成样本的协方差矩阵。

FID度量方式的思想和人类判断是一致的,值域为$[0,+\infty]$,该评价指标值越小,表示生成的图像越接近真实图像,生成的图片质量越好。FID和生成图像的质量之间有很强的负相关性;该度量方式的优势在于其对噪声不是很敏感,而且可以检测出类内的模式崩溃的问题。

\section{GAN研究方向}

\subsection{理论研究}

在理论研究方面,主要的工作是消除生成对抗网络的不稳定性和模式崩溃的问题。Goodfellow,也就是GAN之父,在NIPS会议期间做的一个关于GAN的报告中,阐述了生成模型的重要性,并且解释了生成对抗网络如何工作以及一些前沿的话题。他在生成对抗网络的报告中,主要介绍了几种GAN的网络架构和GAN的应用,并且从信号处理的角度,除了确定训练和构造GAN的方法,还指出了在GAN的理论和实际应用中仍然存在的挑战。此外,损失函数、网络架构、正则化以及批标准化等角度对GAN的一些问题和可重复性等方面也是热门的研究方向,GAN常见的网络结构、训练方法、集成方法、以及一些应用场景都值得探索与研究。

\subsection{应用实践}

在应用方面,主要关注的是生成对抗网络在计算机视觉(CV)、自然语言处理(NLP)和其他领域的应用。目前生成对抗网络在计算机视觉任务中已经有了很多的应用,例如:图像生成、语义分割、图像编辑、超分辨率、图像修复、域转换、视频生成和预测等;而生成对抗网络在自然语言处理中的应用也呈现日益增长的趋势,例如:从文本生成图像、字体的生成、对话生成、机器翻译等;同时生成对抗网络在语音生成方面也有一定应用。在生成对抗网络的众多应用中,被研究最多的领域是图像生成,其目标是通过生成器来生成期望的图像。


\section{GAN的发展及其衍生模型}

研究者们针对GAN存在训练困难等问题,通过不断探索,最终提出了很多基于GAN的变体。下面主要对一些典型的GAN变体进行介绍。

\subsection{DCGAN}
深度卷积生成对抗网络(DCGAN),将卷积神经网络(CNN)应用到生成对抗网络中,通过对GAN的体系结构更改,提高了GAN的训练的稳定性。在DCGAN中,对GAN的体系结构进行了一些修改:将空间池化层函数替换为跨卷积;去除了完全连接层,能提高模型的稳定性;除了生成器的输出层和判别器的输入层之外,对每个单元的输入进行批归一化操作;在生成器中使用$ReLU$激活函数,在其输出层使用$Tanh$函数;在判别器中使用$ReLU$的变体$LeakyReLU$激活函数。DCGAN具有更强大的生成能力,训练也更稳定,生成的样本具有更多的多样性,因此,很多对于GAN的改进都是基于DCGAN的结构。DCGAN只是对GAN模型的结构进行了改进,如\figref{fig:DCGAN生成模型架构图}所示,对生成器和判别器进一步的细化,并没有对优化方法进行改进。

\begin{figure}[!htbp]
  \centering
  \includegraphics[width=0.7\textwidth]{image/DCGAN生成模型架构图.png}
  \caption[]{DCGAN生成模型架构图}
  \label{fig:DCGAN生成模型架构图}
\end{figure}

\subsection{INFOGAN}

GAN强大的学习能力最终可以学习到真实样本的分布,但对输入噪声信号和数据的语义特征之间的对应关系不清楚。一个理想的情况是清楚它们之间的对应关系,这样就能通过控制对应的维度变量来达到相应的变化。比如对于MNIST手写数字识别项目,在知道其对应关系的情况下,可以控制输出图像的光照、笔画粗细、字体倾斜度等。INFOGAN解决了这个问题,它将输入噪声分成两部分,一部分是噪声信号,另一部分是可解释的有隐含意义的信号。INFOGAN模型结构图如\figref{fig:INFOGAN生成模型架构图}所示。

\begin{figure}[!htbp]
  \centering
  \includegraphics[width=0.5\textwidth]{image/INFOGAN生成模型架构图.png}
  \caption[]{INFOGAN生成模型架构图}
  \label{fig:INFOGAN生成模型架构图}
\end{figure}

生成器的输入多了一个隐含变量$C\left(C_{1}, C_{2}, C_{3}, \ldots, C_{L}\right)$,代表的是如上面提到的图像的光照、笔画粗细、字体倾斜度等图像的语义特征信息。INFOGAN对目标函数进行了约束,即$c$和$G(z,c)$之间的互信息,如\eqref{eq:INFOGAN目标函数}:
\begin{equation}
  \min _{G} \max _{D} V_{I}(D, G)=V(D, G)-\lambda I(c ; G(z, c))
  \label{eq:INFOGAN目标函数}
\end{equation}

实验\cite[]{NIPS2016_7c9d0b1f}证明了INFOGAN确实学到了一些可解释的语义特征,通过控制这些特征可以生成想要的数据。如\figref{fig:INFOGAN}通过控制图像的角度、宽度,生成形状不一样的数据。

\figref{fig:INFOGAN_a}在中加入了宽度与倾斜度的信息,INFOGAN中生成的数字中也对应宽度与倾斜度改变;\figref{fig:INFOGAN_b}在中加入了除MNIST数据集手写数字外的其他数据信息,则INFOGAN中生成的数字中展现出了除原始MNITST数据集外的其他特征。INFOGAN具有特征学习的能力,能够提前数据集中的特征。

\begin{figure}[hbtp]
  \centering
  \begin{subfigure}[]{0.38\textwidth}
    \centering
    \includegraphics[width=\textwidth]{image/INFOGAN_a.jpg}
    \caption{在$c$上加上宽度和倾斜度的信息后INFOGAN的生成结果}
    \label{fig:INFOGAN_a}
  \end{subfigure}
  % \hfill
  \quad
  \begin{subfigure}[]{0.38\textwidth}
    \centering
    \includegraphics[width=\textwidth]{image/INFOGAN_b.jpg}
    \caption{在$c$上加上数据类型的信息后INFOGAN的生成结果}
    \label{fig:INFOGAN_b}
  \end{subfigure}
  \caption{图像的角度或宽度的特征学习}
  \label{fig:INFOGAN}
\end{figure}
% \newpage

\subsection{LSGAN}

LSGAN,又叫最小二乘生成对抗网络,LSGAN指出使用JS散度并不能拉近真实分布和生成分布之间的距离,使用最小二乘可以将图像的分布尽可能的接近决策边界,因此LSGAN使用了最小二乘损失函数代替了GAN的损失函数。
该做法有效改善了传统GAN生成图片质量不高,训练不稳定的问题。最小二乘损失与交叉熵损失相比,优势在于生成样本在欺骗判别器的前提下同时让生成器把距离决策边界比较远的生成图片拉向决策边界,这样保证了生成高质量的样本。LSGAN以交叉熵作为损失,会使得生成器不会再优化那些被判别器识别为真实图片的生成图片,即使这些生成图片距离判别器的决策边界仍然很远,也就是距离真实数据比较远,因为此时的交叉熵损失已经很小,生成器完成了为它设计的目标。

\subsection{BigGAN}

Brock等 提出一种新颖的Big-GAN(LargeScaleGAN)模型,训练了迄今为止规模最大的生成性对抗网络,生成的图像十分清晰、逼真、多样,人眼已经难以辨别其真伪了。Big-GAN系统主要改进在以下三方面:
\begin{enumerate}[方面 1.]
  \item 增大batch尺寸

        为了产生逼真、精细的图片,需要提升GAN模型的规模, 如增加batch尺寸和卷积的通道数。生成模型的实验表明,当batch尺寸增加8倍时,生成图像的性能起始分(Inception Score, IS)提高近一倍。分析认为增加batch尺寸引起质量改善可能是每批次覆盖更多模式的结果,为生成网络和判别网络提供了更好的梯度信息。当然,增大batch尺寸会使模型训练的稳定性下降,因而并不能一味地增加batch的尺寸,需要适量考虑。
  \item 采样截断和多层输入

        Big-GAN的采样截断是一种简单的采样技巧,以对生成样本的多样性与保真性进行适当的调节。这种方法通过对$N(0, 1)$分布的输入$z$采样设置圆值的方式来截断采样,其中超出范围的值被重新采样以落入阀值范围内。这个阀值可以根据生成质量、生成多样性等指标决定。Big-GAN采用了分层隐含空间输入技术, 将噪声向量$z$送到$G$的多个层,而不是如普通GAN仅将$z$直接输入到生成网络的初始层。
  \item 模型稳定性的控制

        Big-GAN特有的不稳定性不仅来自G或D, 还来自它们在抗性训练过程的相互作用。为了控制模型的稳定性, 对生成器和判别器增加适当的正则化, 在训练期间监测网络的一系列权重、梯度和损失统计数据, 以寻找可能预示训练崩溃开始的指标。当然, 训练稳定性的获得牺牲了模型的部分性能。

        在128*128分辨率的ImageNet数据库训练后,Big-GAN取得了优秀的成绩,与以往最好的成绩相比较,IS从52.52提升到166.3。\figref{fig:BigGAN生成的图片}中是由Big-GAN生成的图像,可以观察到Big-GAN生成图像的纹理与背景信息十分丰富。

        \begin{figure}[!htbp]
          \centering
          \includegraphics[width=0.7\textwidth]{image/BigGAN_gen.jpg}
          \caption[]{BigGAN生成的图片}
          \label{fig:BigGAN生成的图片}
        \end{figure}
\end{enumerate}


\subsection{LAPGAN}

LAPGAN是一种基于对抗学习的图像超分辨率方法,其主要思想是将图像超分辨率问题转换为生成对抗网络的训练问题,通过对抗学习的方法,训练生成器和判别器,使得生成器能够在给定低分辨率图像的前提下生成高分辨率图像。其训练流程如\figref{fig:LAPGAN模型的学习过程}所示。

LAPGAN的判别器是一个多层的卷积神经网络,用于判断生成器生成的高分辨率图像是否和真实的高分辨率图像相似。LAPGAN的生成器是一个多层的反卷积神经网络,用于将给定的低分辨率图像转换为高分辨率图像。LAPGAN使用了两个损失函数,一个是判别器的损失函数,用于训练判别器,另一个是生成器的损失函数,用于训练生成器。LAPGAN的判别器的损失函数是判别器判断生成的高分辨率图像和真实的高分辨率图像的交叉熵损失,生成器的损失函数是生成器生成的高分辨率图像和真实的高分辨率图像的交叉熵损失。LAPGAN的训练过程是通过不断地训练判别器和生成器,使得生成器和判别器的损失函数达到最优。

\begin{figure}[!htbp]
  \centering
  \includegraphics[width=0.7\textwidth]{image/LAPGAN.png}
  \caption[]{LAPGAN模型的学习过程}
  \label{fig:LAPGAN模型的学习过程}
\end{figure}


\subsection{WGAN}
WGAN的判别器和生成器都使用了卷积神经网络,判别器使用卷积神经网络提取图像特征,然后将特征输入全连接层,最后使用激活函数输出判别结果,生成器也是使用卷积神经网络,输入的随机噪声经过卷积层逐渐变大,最后输出生成的图片。WGAN的判别器使用的是一个线性层,而不是传统的sigmoid层,判别器的输出结果是一个数值,而不是一个概率,因此判别器的输出结果不需要通过sigmoid函数转换为概率,也不需要通过对数函数转换为对数概率。WGAN的判别器和生成器都没有使用激活函数,因为激活函数会使得生成器和判别器的梯度消失,从而影响训练效果。WGAN的判别器和生成器都没有使用全连接层,因为全连接层会导致生成图片的像素之间的相关性变强,导致生成的图片失真。WGAN的判别器和生成器都没有使用池化层,因为池化层会使得生成图片的像素之间的相关性变强,导致生成的图片失真。WGAN的判别器和生成器都没有使用BN层,因为BN层会使得生成图片的像素之间的相关性变强,导致生成的图片失真。

\subsection{StackGAN}

根据文字描述人工生成对应高质量图片一直是计算机视觉领域的一个挑战,StackGAN是第一个
根据文字描述生成图像分辨率达到256*256的网络模型。其网络架构如\figref{fig:StackGAN模型架构}所示。

\begin{figure}[!htbp]
  \centering
  \includegraphics[width=\textwidth]{image/StackGAN.jpg}
  \caption[]{StackGAN}
  \label{fig:StackGAN模型架构}
\end{figure}

StackGAN的训练分为两个阶段,它们分别被定义为Stage-\uppercase\expandafter{\romannumeral1}和Stage-\uppercase\expandafter{\romannumeral2}。Stage-\uppercase\expandafter{\romannumeral1}阶段是根据给定的文字描述,学习到初始的形状和色彩,生成低分辨率(64*64)的图片;Stage-\uppercase\expandafter{\romannumeral2}阶段根据Stage-\uppercase\expandafter{\romannumeral1}生成的低分辨率图像以及原始文字描述,生成具有更多细节的高分辨率(256*256)图片。在Stage-\uppercase\expandafter{\romannumeral1}阶段中,词嵌入向量$\varphi_{t}$经过全连接层生成高斯分布$N(\mu_0(\varphi_t),\sigma_0(\varphi_t))$中的 $\mu_0$和$\sigma_0$,最后与$z$噪声向量拼接作为$G_0$的输入,通过一组上采样生成64*64的图像。Stage-\uppercase\expandafter{\romannumeral2}阶段以高斯隐含变量$\hat{c}$以及Stage-\uppercase\expandafter{\romannumeral1}生成器的输出作为输入,来训练生成器和判别器。但是这种分段式模式可能会出现每个任务找不到重点,最终导致生成失败。

StackGAN 及其改进版StackGAN++都是用于文本生成图像的网络,目前在该领域比较新的突破性工作是由斯坦福大学李飞飞教授实验室发表的文献。李飞飞教授的实验室就是著名的图像数据集ImageNet项目的组织者。该研究引入了场景图的概念,相比于StackGAN和StackGAN++方法,它能够处理更加复杂的文本,并且取得了不错的效果。

\begin{figure}[!htbp]
  \centering
  \includegraphics[width=0.7\textwidth]{image/StackGAN++.jpg}
  \caption[]{StackGAN}
  \label{fig:StackGAN++模型架构}
\end{figure}

\subsection{BeGAN}

传统的GAN是优化生成数据分布与真实数据分布之间的距离,认为两者的分布越相似,$G$的生成能力越好,很多GAN的改进也是基于该思路。BEGAN(boundary equilibrium GAN)并没有直接去估计生成分布$p_g$和真实分布$p_x$的距离,而是估计两者分布误差的距离。如果分布之间的误差分布相近的话,也可以认为$p_g$和$p_x$是相近的。BEGAN将判别器$D$设计成自编码器用来重构分布误差,并优化分布误差之间的距离,如下式:

\begin{equation}
  \left\{\begin{array}{l}
    L(x, D)=|x-D(x)| \\
    L(G(z), D)=|G(z)-D(G(z))|
  \end{array}\right.
\end{equation}

BEGAN提出一种均衡概念,用以平衡$G$和$D$的训练,使GAN即使使用很简单的网络,不加如BN、minibath等训练技巧也能得到很好的训练效果。同时还提出了一种能够在样本多样性和样本质量上均衡的超参数以及衡量模型收敛性的方法。实验中发现BEGAN收敛很快,并且$G$和$D$训练平衡,但超参数的选取比较考量经验。




\section{GAN的各类应用场景}

计算机视觉是目前人工智能研究的重要领域,而GAN在计算机视觉的许多方面都表现非凡,从最初的图像生成,到后面的一系列应用,越来越多新的GAN框架被提出并应用到新的领域,由于GAN自身的对抗特性它能不断地自我提升,在生成样本领域取得了比传统方法更显著的效果。下面将介绍GAN在计算机视觉上的应用及为了实现目标任务在结构上做出的改变。

\subsection{图像分类}

GAN网络必定包含一个判别器,实际上就是一个天生的分类器,再加上生成器的辅助,实现的是对原始图像和生成的图像的联合分类,而不是对图像单独分类,使得GAN可以完成比一般统计分类更加复杂的分类任务。GAN能够生成和原始图像分布相同的模拟样本,在分类过程中,通过添加这些模拟样本,等价于人为增加训练样本集的数量,有助于提高分类器的判别能力。

例如,使用适当数量的“图像标签”对,加上简单的线性模型,判别器的卷积层输出可用作特征提取器。在一个DC—GAN网络内,可以用正则化L2—SVM分类器对训练后判别器中提取出的特征矢量来进行评价。使用这种方法,在全监督和半监督数据集上已取得良好的分类效果。当有标注训练数据提供有限时,对抗训练还可以用于合成更多的训练数据。

如果使用GAN来提炼合成图像,同时维持它们的标注信息,通过仅在GAN提炼的合成图像(即不是真实的图像)上进行训练,在姿态和注视估计(pose—and gaze-estimation)实验中取得了优异的性能。类似地,使用时空GAN结构,注视估计和预测也获得了很好的结果。在某些实际应用情况下,合成数据上训练的模型并不能很好地泛化。但如果使用对抗训练的调适方法,将来自源域的合成样本匹配到目标域。

\subsection{图像超分辨率}

图像超分辨率是指由一幅低分辨率图像或图像序列恢复出高分辨率图像,此技术分为超分辨率复原和超分辨率重建。
图像超分辨率一直是计算机视觉领域的一个研究热点,SRGAN是GAN的一个变体,也是GAN在图像超分辨率应用上的一个成功案例。SRGAN基于相似性感知方法提出了一种新的损失函数,有效解决了恢复后图像丢失高频细节的问题,并使人能有良好视觉感受。SRGAN从特征上定义损失,它将生成样本和真实样本分别输入VGG-19网络,生成网络包含5个残差块,每个残差块包含两个3x3、64特征的卷积层,后接BN层,激活函数选择$PReLU$。残差块后面两个2倍的分步幅卷积层用来增大特征尺寸。判别网络部分包含8个卷积层,随着网络层数加深,特征个数不断增加,特征尺寸不断减小,最终通过两个全连接层和一个$sigmoid$激活函数得到判断为自然图像的概率。
SRGAN的损失函数为感知损失函数,它包括加权的两部分,一部分是内容损失,另一部分是对抗损失函数。内容损失函数采用特征空间的最小均方差来表示。对抗损失函数和判别器输出的概率有关。
\figref{fig:超分辨率图像生成}是应用在图像超分辨率上的网络模型实验的结果,由于SRGAN的网络架构与损失函数便于优化,生成的超分辨率图像明显优于其他模型。

\begin{figure}[!htbp]
  \centering
  \includegraphics[width=0.7\textwidth]{image/超分辨率图像生成.png}
  \caption[]{超分辨率图像生成}
  \label{fig:超分辨率图像生成}
\end{figure}

\subsection{风格迁移}

图像的风格迁移就是将一幅图像转换为另一种风格的图像,如将一张照片图像经过GAN处理,自动转化为油画风格的图像。深度学习最早是使用CNN框架来实现的,但这样的模型存在训练速度慢,对训练样本要求过高等问题。由于GAN的自主学习和生成随机样本的优势,以及降低了对训练样本的要求,使得GAN在图像风格迁移领域取得了丰硕的研究成果。

如果输入和输出的训练图像是同一场景,只是表现风格不同,这是一种\textbf{配对训练}。GAN的变种CycleGAN利用循环GAN进行无监督地从一种风格图像转化到另一种风格的图像,是一种无配对的训练方法:输入和输出的训练图像不仅风格不同,而且内容也不同。这种方法比配对训练方法难度更大,但训练数据的来源也大大扩展了。如下图中将照片风格转换为不同艺术家的油画风格。

\begin{figure}[!htbp]
  \centering
  \includegraphics[width=0.7\textwidth]{image/风格迁移.png}
  \caption[]{风格迁移}
  \label{fig:风格迁移}
\end{figure}

但是CycleGAN仍然存在一些问题,例如在将马转化为斑马的时候,很可能会将骑马的人也加上斑纹。因为原始训练集都是大自然中的马,因此模型学习不到人的相关知识。并且发现有些时候模型会将一些比较突出的背景也加上斑纹,如石头,醒目的植物等等,因此在稳定性方面还是有提升空间的。此外,CycleGAN不易实现形状的改变,比如将猫改成狗,外表形状上必然要发生变化,并且CycleGAN不易训练,训练周期较长,这也是可以改进之处。

\subsection{图像修复}

图像修复是图像处理中的一个经典的应用领域,传统的图像修复方法通常是根据周围的像素点估计待修复的像素点,但是这种算法大多比较复杂,而且对于大面积的图像损毁很难修复。由于GAN是通过对抗博弈的方式来进行训练的,所以在图像修复方面不用受限于可用的图像统计信息,且能使得修复效果更加自然。
SNGAN提出了一种生成式图像修复系统,可以使用自由形式的掩模和输入来完成图像。该系统基于从数百万张图像中学习的卷积,无需额外的标记工作,解决了将所有输入像素都视为有效像素的问题,使训练快速稳定。但由于GAN模型本身存在的问题,导致在修复图像时,可能会出现过度平滑或模糊的情况。
GAN技术的引入提高了修复后图像的质量。在各种有损图像的修复中,人脸图像的修复是一种相对比较困难,但有广泛应用需求的技术。GAN用作人脸补全,主要由一个生成器、两个判别器和一个解析网络组成。生成器实际上是一个自编码器,输入为有损的整个脸部图像。一个判别器鉴别输入的整个真实图像和生成图像的真伪,另一个判别器鉴别输入的遮挡部分的生成图像和真实图像的真伪。生成图像的非遮挡部分用真实图像的对应像素替代。解析网络将脸部分割为不同个语义部位(如嘴、鼻、眼),确定每个像素所属的部位。这种方式可以同时获取补全后的完整图像和补全部分的信息,避免模型出现仅仅关注补全那一部分时带来的误判。\figref{fig:人脸补全GAN网络结构}是人脸补全GAN网络结构:

\begin{figure}[!htbp]
  \centering
  \includegraphics[width=0.8\textwidth]{image/人脸补全GAN网络结构.png}
  \caption[]{人脸补全GAN网络结构}
  \label{fig:人脸补全GAN网络结构}
\end{figure}

它基于神经网络直接生成缺失区域的内容,通过引入重建损失,两个对抗性损失和语义解析损失的组合进行训练,确保了像素忠实度和局部全局内容的一致性。它能处理任意形状的大面积缺失像素,并产生逼真的面部完成结果。在人脸图像数据集FFHQ上的实验展示了这种方法对关键部位缺失的人脸图像的高质量补全的结果。如\figref{fig:人脸补全效果}所示。

\begin{figure}[!htbp]
  \centering
  \includegraphics[width=0.7\textwidth]{image/人脸补全效果.png}
  \caption[]{人脸补全效果}
  \label{fig:人脸补全效果}
\end{figure}

\subsection{视频预测}

由于基于生成对抗网络做图像生成可以保持图像的细节纹理特征。由于自然场景的内容复杂和运动变化,对视频中将来帧(future frame)的预测是一项具有挑战性的任务,也是无监督视频表示学习的关键技术之一。现存的预测方法致力于对像素值的直接估计,和实际的将来帧有一定的差距,容易形成模糊的预测结果。
图像模糊是视觉任务中经常遇到的一个问题,比如:图像数据采集过程中由于物体运动导致的模糊,目标跟踪中相机的运动导致的模糊等。
dual motion GAN是一种双重运动 GAN结构。为了使合成的将来的视频帧与真实的视频帧没有区别,这种GAN通过双重对抗训练机制,迫使预测的将来帧和视频中像素流(保持一致。为了更好地 预测,最初的将来帧预测和双重的将来流预测形成一个闭环,相互之间传递反馈信号。

\begin{figure}[!htbp]
  \centering
  \includegraphics[width=0.4\textwidth]{image/视频预测的关键帧处理.png}
  \caption[]{视频预测的关键帧处理}
  \label{fig:视频预测的关键帧处理}
\end{figure}

\subsection{3维深度图像生成}

现实世界是3D的,3维深度图像无疑能够更加真实地反映现实世界和人类视觉感受。2D视觉生成不可避免地在许多领域的实际应用中受到限制,如人脸3D建模、机器人学习、虚拟现实、游戏行业和设计行业等。3维深度图像生成的关键问题在于如何从2D图像或文本等数据中构建出深度信息并进行真实准确的3维建模。

视角合成是实现3维深度图像生成的一种重要方法,其通过已有的一组视角生成未知的目标视角。基于多视角图像可以进行3维建模。使用GAN针对视角合成中像素匹配难度大、生成图像质量差的问题,采用自洽机制与已有视角进行一致性约束,进而结合生成模型生成高质量的目标视角图像。实现了从单幅2D图像重构3维深度图像并取得了很好的效果,仅使用一幅单视图图像而无需额外监督信息来生成高质量的3维物体模型。该方法基于对称性假设,采用多个encoder-decoder网络将一幅物体图像分解为深度、光照和视角等多个维度,组合渲染,重构出3维物体模型。

\begin{figure}[!htbp]
  \centering
  \includegraphics[width=0.6\textwidth]{image/房屋三维建模.jpg}
  \caption[]{房屋三维建模}
  \label{fig:房屋三维建模}
\end{figure}

\section{面临的挑战和未来的研究方向}

\subsection{GAN面临的挑战}

\begin{enumerate}[挑战 1.]
  \item 性能与成本权衡的问题。

        SAGAN、BigGAN 等都证明了庞大的GAN模型能得到性能上的提升,如BigGAN生成了超清晰的图像,但是庞大的模型需要使用TPU或者多个GPU训练,成本高。一个可行的方法是保证性能的同时将模型压缩,如使用模型量化等方法,但这仍是一个严峻的挑战。

  \item 生成器的稳定性问题。

        GAN在使用类别特定数据集(如人脸、卧室)进行训练时已经能生成逼真、高质量样本,但建模具有多个类别,高度多样化的数据集(如ImageNet)的分布仍然是一项重大的挑战,通常需要根据另一种输入信号来调节生成或为特定任务训练模型,此外GAN在这种高可变性的数据集中难以生成全局一致的高分辨率样本。

  \item 生成器的可解释性问题。

        GAN的全局收敛性未得到真正证明。GAN的训练方式是交替优化生成器和判别器,这和其他神经网络不同。目前比较常用的是基于博弈理论原理进行GAN的训练,但这一过程是在资源限制的情况下完成的,下一步应该要研究怎样减少这样的资源限制。或者试图分析常规神经网络来解答GAN收敛性的问题,因为生成器模型和判别器模型的参数均为非凸性损失函数,这与绝大多数神经网络有相似之处。
\end{enumerate}

\subsection{GAN研究方向进展}

\begin{enumerate}[方向 1.]
  \item 理论突破。
        GAN的理论基础是博弈论中的二人零和博弈,目前还没能在理论上真正证明均衡点的存在性,同时GAN训练的不稳定性如梯度消失、模式崩溃、过拟合和生成样本自由度较高等模型性能问题仍是当前GAN所面临的问题。
  \item 算法扩展。
        GANs在图像合成、图像超分辨率等连续样本上表现良好,但是在离散样本如文本处理上的表现还低于基于似然的语言模型。对GAN的算法拓展,使GAN这一个优秀模型的应用范围更加广泛,是迫切需要解决的问题。目前的一个方向是GAN吸收机器学习中最新的理论与研究成果并与之相结合,如GAN与强化学习结合。
  \item 解决对抗样本带来的困扰。
        分类器极易受到对抗样本的影响,往往会因为故意添加的一些细微干扰就会导致模型判断错误。比如判别器$D$判别一个生成样本$G(z)$为假,却将加了扰动的生成样本$G(z)+p$判定为真。对抗样本影响GAN的性能,如何解决这个问题,使得模型的精度和稳定性有所提升,仍值得探索。
\end{enumerate}

\section{结语}

本文对基本GAN模型的结构进行了介绍,并分析了GAN的优缺点与评价方法。其次,总结介绍了部分GAN的衍生模型。

生成对抗网络GAN由生成网络$G$与判别网络$D$组成,有很强的生成能力,主要应用在计算机视觉领域。现在GAN可以产生超分辨率图像、完成图像的风格迁移、进行图片补全、预测视频关键帧等等。GAN为人工智能注入了活力,特别是计算机视觉领域,其无监督学习的生成对抗方法极大促进了该领域的发展。它作为一种生成模型,对于解决样本不足、生成质量差、提取特征难度大等问题提供了一种较好的解决方案。对基于深度学习的生成对抗网络在计算机视觉方面的应用进行了分析总结,不仅深入分析了GAN在理论模型方面的改进,而且重点介绍了GAN在视觉方面的几类突出的应用。

GAN的训练类似于人类一理解周围复杂的世界的过程,其对抗生成的思想与人的思维模式异曲同工。在GAN这个方向继续探索,或与有可能创造出具有认知的机器学习模型,在其奥妙之中,有着许多机会进行理论探索和技术开发,其中也蕴藏着大量创新引用的机会。

\section{致谢}

十分感谢在百忙之中抽出时间审阅本文的老师。也因为《研究方法与前沿(计算机)》这门课,我有机会探索神秘而又有趣的GAN。GAN绝对是当今计算机视觉的热门领域,有大量的学者和研究人员在GAN领域进行探索。也正是这样一个机会,我可以从\href{https://www.ccf.org.cn/c/2019-04-25/663625.shtml}{CCF评级}中的许多A类期刊中,例如\href{https://cvpr2022.thecvf.com/}{CVPR}、\href{https://www.aaai.org/Conferences/AAAI-22/}{AAAI}、\href{https://www.ijcai.org/}{IJCAI}、\href{https://www.ijcai.org/}{ICCV}等等,找到了许多有趣的GAN的应用。在这里,我也要感谢老师的指导,让我有机会在这个领域中进行探索。由于本人的学识和写作的水平有限,在本文的写作中难免有僻陋,恳请老师多指教。

% 参考文献
\newpage
\nocite{*}
\printbibliography[heading=bibintoc, title=\ebibname]

% 附录
\appendix
% \appendixpage
\addappheadtotoc

\end{document}



