%!TEX program = xelatex
% 完整编译: xelatex -> biber/bibtex -> xelatex -> xelatex
\documentclass[lang=cn,a4paper,chinesefont=founder,bibend=bibtex]{shu-lab-report}
% * 更多配置选项
% \documentclass[lang=cn,a4paper,zihao=-4,chinesefont=founder,bibend=bibtex]{elegantpaper}

% 本文档命令
\usepackage{array}

\newcommand{\ccr}[1]{\makecell{{\color{#1}\rule{1cm}{1cm}}}}
\renewcommand{\lstlistingname}{代码}


\begin{document}

%%%%%%%%%%%%%%%%%%%%%%%%%%%%%
%% The Cover Page
%%%%%%%%%%%%%%%%%%%%%%%%%%%%%

\thispagestyle{empty}

~\\

\vspace{1.5cm}

\begin{figure}[!htbp]
  \centering
  \includegraphics[width=10cm]{image/shulogo.png}
\end{figure}

\centerline{\large{\textbf{SHANGHAI  UNIVERSITY}}}

\vspace{8mm}

\centerline{\kaishu\Huge{\textbf{实验\hspace*{1em} 多核环境下OpenMP并行编程}}}

\vspace{16mm}

\begin{center}
  \renewcommand\arraystretch{2}
  \begin{tabular}{r c}
    \makebox[8em][s]{\LARGE{组号}}   & \LARGE{5}           \\         \cmidrule(l){2-2}
    \makebox[8em][s]{\LARGE{学号姓名}} & \LARGE{20121034胡才郁} \\  \cmidrule(l){2-2}
    \makebox[8em][s]{\LARGE{实验序号}} & \LARGE{3}           \\            \cmidrule(l){2-2}
    \makebox[8em][s]{\LARGE{日期}}   & \LARGE{2023年5月9日}   \\ \cmidrule(l){2-2}
  \end{tabular}
\end{center}

\vspace{16mm}

% Please add the following required packages to your document preamble:
% Note: It may be necessary to compile the document several times to get a multi-page table to line up properly
{\Large
  \begin{longtable}[c]{|c|c|}
    \hline
    \textbf{评分内容}                 & \textbf{评分} \\ \hline
    \endhead
    %
    内容完整,有实验目的、步骤、分析、总结           &             \\ \hline
    内容质量,表述清楚正确,条理清晰,分析总结到位       &             \\ \hline
    格式规范,标题、段落、公式、图表、代码符合专业文献出版要求 &             \\ \hline
    综合得分                          &             \\ \hline
  \end{longtable}
}

\newpage

%%%%%%%%%%%%%%%%%%%%%%%%%%%%%
%% The Content
%%%%%%%%%%%%%%%%%%%%%%%%%%%%%

\begin{tcolorbox}[title=要求1]
  OpenMP介绍
\end{tcolorbox}

OpenMP的主要目标是使并行编程变得容易且高效,通过在代码中添加简单的编译指令来进行并行化。这些指令使用特殊的语法标记来标识哪些代码段可以并行执行,因此程序员可以针对性地对代码进行并行化,而不需要全面地重构整个应用程序。

本次实验在本人虚拟机上的Ubuntu操作系统上完成,具体而言,OpenMp的安装可以通过以下步骤进行:

\begin{enumerate}
  \item 输入以下命令以安装OpenMP库:
        \begin{lstlisting}[language=bash]
  sudo apt-get install libomp-dev
  \end{lstlisting}

  \item 安装成功后,在gcc编译时加入\codeinline{-fopenmp}参数即可使用OpenMP库。

\end{enumerate}



\begin{tcolorbox}[title=要求2]
  创建多线程,输出线程号和线程数。
\end{tcolorbox}

OpenMP提供\codeinline{omp\_set\_num\_threads()}函数可以设置程序的线程数,而\codeinline{omp\_get\_thread\_num()}函数可以获取当前线程的编号,\codeinline{omp\_get\_num\_threads()}函数可以获取当前线程的总数。通过这些函数,可以很方便地实现多线程的创建。在下面的代码中,通过tid来获取线程编号,并且指定tid为0的线程输出线程总数。具体代码如下:

\begin{lstlisting}[language=c]
#include <omp.h>
#include <stdio.h>
int main()
{
    int nthreads, tid;
    omp_set_num_threads(16);
#   pragma omp parallel private(nthreads, tid)
    {
        tid = omp_get_thread_num();
        printf("Hello World from OMP thread %d\n", tid);
        if (tid == 0)
        {
            nthreads = omp_get_num_threads();
            printf("Number of threads is %d\n", nthreads);
        }
    }
}
\end{lstlisting}

编译运行结果如\figref{fig:1}所示, 可以观察到,在给定16个线程的情况下,线程编号从0到15,且线程0输出了线程总数。并且输出的顺序并非按照线程编号的顺序,这是因为线程的创建和执行是并行的,因此输出的顺序是不确定的。

\newpage

\begin{figure}
  \centering
  \includegraphics[width=0.3\textwidth]{image/1.jpg}
  \caption{多线程输出线程号和线程数}
  \label{fig:1}
\end{figure}


\begin{tcolorbox}[title=要求3]
  学习for多线程并行
\end{tcolorbox}

下面的代码给出了使用for多线程并行计算数组元素和的简单实现,利用\codeinline{reduction}规约完成,更复杂的矩阵乘法实现在后面在报告后面的部分给出。

首先定义一个数组a,包含100个元素,然后使用循环将数组中的元素依次初始化为1,2,3,...,100。接下来使用OpenMP并行计算数组元素的总和。使用\codeinline{\#pragma omp parallel for}指令将for循环变成一个并行计算的任务,其中第二个参数\codeinline{reduction(+:sum)}告诉OpenMP对sum变量进行求和操作,并使用重复加法算法保证线程安全。

其中对于线程安全的实现主要通过\codeinline{reduction}完成,它可以帮助我们实现并行任务中的归约操作。在并行任务中,某些变量需要被多个线程访问和修改,这时就需要对这些变量进行归约,即将多个线程对变量的计算结果合并成一个最终结果。

\begin{lstlisting}[language=c]
#include <stdio.h>
#include <omp.h>

int main() {
    int i, n = 100, sum = 0;
    int a[n];
    // 初始化数组
    for (i = 0; i < n; i++) {
        a[i] = i + 1;
    }
    // 并行计算数组元素的总和
    #pragma omp parallel for reduction(+:sum)
    for (i = 0; i < n; i++) {
        sum += a[i];
    }
    printf("The sum is: %d\n", sum);
    return 0;
}
\end{lstlisting}

结果如\figref{fig:4}所示,正确的求出了1-100的和,即5050,说明了\codeinline{reduction}规约的正确性。

\begin{figure}
  \centering
  \includegraphics[width=0.3\textwidth]{image/4.jpg}
  \caption{for多线程并行计算数组元素和}
  \label{fig:4}
\end{figure}

\begin{tcolorbox}[title=要求4]
  学习while多线程并行,实现全局共享变量存取
\end{tcolorbox}

如下,实现了使用while多线程并行。下面的代码中的全局共享变量是 i,该变量被所有线程共享,并且在 while 循环中被多个线程同时访问和修改。

每个线程在访问并修改 i 变量之前必须确保其他线程不同时访问和修改相同的变量 i 。

使用了 \codeinline{\#pragma omp atomic} 将 i 变量的递增操作转换为原子操作。这将保证 i 变量被原子地递增,从而避免多个线程同时读取和增加 i 变量的值所带来的问题。最后,在同步点处同步 i 的值,使用了 \codeinline{\#pragma omp flush(i)} 指令,它确保对 i 进行可见性刷新,这个过程确保任何一个线程都可以读取到前一个线程写入的新值。

\begin{lstlisting}[language=c]
#pragma omp parallel private(tid)
{
  tid = omp_get_thread_num();
  #pragma omp critical
  printf("Thread %d is running\n",tid);

  #pragma omp barrier

  #pragma omp while
  while(i < n) 
  {
      #pragma omp atomic
      i++;
      #pragma omp critical
      printf("Thread %d says i=%d\n",tid,i);

      #pragma omp flush(i)
  }
}
\end{lstlisting}

程序运行结果如\figref{fig:2}所示。可以看出,完成了while在多线程并发的使用,输出时,线程的顺序是不确定的,但是输出的i的值是递增的,这说明了while的多线程并行是成功的。

\newpage

\begin{figure}
  \centering
  \includegraphics[width=0.3\textwidth]{image/2.jpg}
  \caption{while多线程并行}
  \label{fig:2}
\end{figure}

\begin{tcolorbox}[title=要求5]
  编程实现大规模向量的并行计算
\end{tcolorbox}

以下代码为使用OpenMP实现矩阵之间乘法的核心代码块。调用了\codeinline{omp\_get\_wtime()}包围了核心代码块,以便计算程序运行时间。在核心代码块中,使用了\codeinline{omp\_parallel}和\codeinline{omp\_for}来实现多线程并行,其中\codeinline{omp\_parallel}用于创建线程,\codeinline{omp\_for}用于指定需要并行的代码块。

在\codeinline{omp\_for}中,通过\codeinline{private}来限制变量的作用域,从而避免了多线程对同一变量的读写冲突。具体而言,\codeinline{i}、\codeinline{j}和\codeinline{k}是循环变量,因此需要在每个线程中单独创建。在计算矩阵乘法时,将矩阵的每一行分配给不同的线程进行计算。对于每一个进程而言,其任务是将矩阵A的第i行与矩阵B的第j列相乘,然后将结果累加到矩阵C的第i行第j列。因此,在每一个线程的局部计算结束后,便得到了矩阵C的一部分,最后将这些部分合并即可得到最终结果。下述代码以8线程为例,具体代码如下:

\begin{lstlisting}[language=c]
// 8线程并行
start = omp_get_wtime();

#pragma omp parallel num_threads(8)
{
#pragma omp for private(i, j, k)
    for (i = 0; i < N; i++)
    {
        for (j = 0; j < N; j++)
        {
            *(c + i * N + j) = 0;

            for (k = 0; k < N; k++)
            {
                *(c + i * N + j) += *(a + i * N + k) * *(b + k * N + j);
            }
        }
    }
}

end = omp_get_wtime();
cpu_time_used = end - start;
ratio = serial_time_used / cpu_time_used;

printf("8线程并行计算时间:%f, 加速比:%f\n", cpu_time_used, ratio);
\end{lstlisting}

实验结果如\figref{fig:3}所示。观察结果可以发现,随着线程数的增加,在线程数量较少时,加速比可以看成线性增长,而档线程数为6、8、16时,加速比接近4.5,不再增长。出现这样的实验结果,是由于我分配给这台实验机器的虚拟CPU核心数为6,因此当线程数接近或者超过6时,在计算任务较长,CPU核心满载时,线程数的增加并不能带来更多的加速比,只会趋近于理论最大加速比6。

\begin{figure}[htbp]
  \centering
  \includegraphics[width=0.7\textwidth]{image/3.jpg}
  \caption{矩阵乘法计算实验结果}
  \label{fig:3}
\end{figure}

\end{document}